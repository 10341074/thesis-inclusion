\iffalse
\section{Tools}
\begin{theorem}[Banach-Steinhaus or Uniform Boundedness]
 $X$ B-sp, $Y$ n.v.s. Let $\mathcal{F} \subset \mathcal{L}(X,Y)$: pointwise boundedness implies uniform boundedness of $\mathcal{F}$:
 \[ \forall x \quad \exists M_x\geq 0 \quad s.t. \quad \sup_{F \in \mathcal{F}} \| Fx\|_Y \leq M_x\]
 \[ \Rightarrow \quad \exists M>0 \quad s.t.\quad \sup_{F\in\mathcal{F}}\| F\|_{\mathcal{L}(X,Y)} \leq M\]
\end{theorem}
\begin{remark}
 If hypothesis doesn't hold, $\exists G_\delta$ such that  
\end{remark}
\begin{proposition}
 Under same hypotheses, given a sequence $\{F_n\}_n \subset \mathcal{L}(X,Y)$:
 \[\forall x: \quad F_n x\to y=:Fx\quad \Rightarrow \quad F \in \mathcal{L}(X,Y)
 \]
\end{proposition}

\begin{theorem}[Open Mapping]
 $X,Y$ B-sp. If $F\in\mathcal{L}(X,Y)$ is surjective, then $F$ is an open map
\end{theorem}
\begin{theorem}[Inverse Mapping]
 $X,Y$ B-sp. If $F\in\mathcal{L}(X,Y)$ is bijective, then $F$ has linear continuous inverse $F^{-1}\in \mathcal{L}(X,Y)$ 
\end{theorem}
\section{Introduction}
\begin{definition}[Well-Posedness]
 Let $T:U\subset X\to Y$, the equation $Tx=y$ is well-posed if the following properties hold:
 \begin{enumerate}
  \item Existence: surjectivity
  \item Uniqueness: injectivity
  \item Continuous depandance on data: continuous inverse
 \end{enumerate}
\end{definition}
Not-uniqueness means \textit{lack of information}. Even instability can be solved only with \textit{further information} 
\begin{definition}
 A \textbf{compact} operator maps bounded sets in relatively compact sets
\end{definition}
\begin{definition}
 A \textbf{completely continuous} operator is a compact continuous operator
\end{definition}
\begin{remark}
 If $K$ is a completely continuous operator, than the equation of first kind $Kx=y$ is ill-posed 
 if $U$ is of infinite dimension
\end{remark}
\begin{definition}
 Let $X_1\subset X$ a subspace with stronger norm $\|\cdot\|_1$, the \textbf{worst-case error} is:
  \[\mathcal{F}(\delta,E,\|\cdot\|_1)=\sup \{\|x\| : x\in X_1, \|x\|_1 \leq E, \| Tx\|\leq \delta\}\]
\end{definition}
The aim is fixed $E$: $\mathcal{F}(\delta, E, \| \cdot \|_1)$ for $\delta \to 0$

\section{Regularization}
In general to solve $Kx=y$, for $X$ quotient a non-trivial kernel, is not restrictive assuming:
\begin{center}
 $K:X\to Y$ is a linear compact, injective, (with dense range in $Y$) operator \\ $X,\,Y$ are H-sp (or B-sp)
\end{center}
\begin{remark}
 $\mathcal{R}(K)$ is a dense NOT complete subspace in $Y$, due to the fact that Banach-Steinhaus theorem for $K^{-1}$ leads to a contradiction in the opposite case
\end{remark}
The best we can hope is to recover asymptotically the \textit{worst-case error} estimate speed $\mathcal{F}(\delta,E,\| \cdot \|_1)$ under a priori information on exact solution $\| x \|_1\leq E$
\begin{definition}
 A \textbf{regularization strategy} is a family $\{R_\alpha\}_\alpha$ of continuous, pointwise convergent operators on $\mathcal{R}(K)$ (or equivalently $\{R_\alpha K\}_\alpha$ on $X$)
\end{definition}
\begin{remark}
 Some facts:
 \begin{itemize}
  \item[-] $\{R_\alpha\}_\alpha$ are NOT uniformly bounded
  \item[-] $\{KR_\alpha\}_\alpha$ are uniformly bounded (Banach-Steinhaus) but NOT norm $\mathcal{L}(X)$ convergent to $I$ (i.e. uniformly pointwise convergent on bounded setx in $X$)
 \end{itemize}
\end{remark}
Competition between stability vs accuracy:
\[\|x - x_\alpha^\delta \| \leq \| R_\alpha\|\delta + \| R_\alpha K x - x\| \]
A regularization strategy is the choice $\alpha(\delta)$
\begin{definition}
 A regularization strategy is \textbf{admissible} if $\alpha(\delta) \to 0$ and error $e(x,x_\alpha^\delta) \to 0$ for  $\delta\to 0$
\end{definition}
A regularization example is filtering singular system $(x_j,y_j,\mu_j)$ for a linear compact operator $K:X\to Y$
\section{Appendix: complex calculus}
We define $\| a \|^2 = (a, \overline{a})$ and use antilinearity $(a,b) = \overline{ (a,b) }$
\[(ix,iy)=-i^2(x,y)=(x,y)\]
\[ ( a,\overline{b} ) + (b,\overline{a}) = 2 (\Re(a)), \Re(b)) - 2(\Im(a),\Im(b)) = 2\Re((a,b)) = 2\Re((b,a))\]
\section{Results}
\begin{theorem}[Green Representation Formula]
 $D$ of class $C^1$, $u\in C^2(D)\cap C^1(\overline{D})$:
 \begin{equation}
  u(x) = -\int_D\Delta u(y) \Phi(x,y) dy + \int_{\partial D}(u_\nu(y)\Phi(x,y)-\Phi_\nu(x,y) u(y)) dy
 \end{equation}
\end{theorem}
\begin{theorem}[Reciprocity Relation]
 Consider two problems with $0 < \gamma_0 \leq \gamma(x) \in L^\infty(\mathbb{R}^2)$:
 \begin{equation}
  \divergence(\gamma \nabla u_j(\cdot,x_j)) = - \delta_{x_j}\quad j = 1,2
 \end{equation}
\end{theorem}
\begin{proof}
 Apply Green formula to $\gamma$
\end{proof}
\fi


\iffalse
\section{The direct and inverse problem}
\subsection{Fundamental Problem}
\begin{definition}
 Define the fundamental solution of homogeneous laplace problem in $\mathbb{R}^n$:
 \begin{equation}
  \Delta \Phi(x,x_0) = -\delta_{x_0} \;\textup{in}\;\mathcal{D}'(\mathbb{R}^n)
 \end{equation}
 \begin{equation}
  \Phi(x,x_0)=
  \begin{cases}
   \frac{1}{4\pi}\frac{1}{| x |} & n=3 \\
   \frac{1}{2\pi}\log(\frac{1}{|x|}) & n=2
  \end{cases}
 \end{equation}

\end{definition}

\begin{definition}[Fundamental Solution Problem]
 Define the solution named Green function of the following equation in distributional sense:
 \begin{equation}
  \divergence((1+(k-1)\chi_D)\nabla G(x,x_0)) = - \delta_{x_0}\;\textup{in}\;\mathcal{D}'(\mathbb{R}^n)
 \end{equation}
 imposing vanishing at infinity for $n=3$, imposing logaritmic growth for $n=2$ (boundedness condition for exterior domain $D_e$ problem ?)
\end{definition}
Let $G(x,x_0)=\Phi(x,x_0)+u(x,x_0)$:
\begin{equation*}
  \divergence((1+(k-1)\chi_D)\nabla G(x,x_0)) = \divergence(\nabla \Phi(x,x_0)) \;\textup{in}\;\mathcal{D}'(\mathbb{R}^n)
\end{equation*}
\begin{equation}
  \divergence((1+(k-1)\chi_D)\nabla u(x,x_0)) = -\divergence((k-1)\chi_D \nabla \Phi(x,x_0)) \;\textup{in}\;\mathcal{D}'(\mathbb{R}^n)
\end{equation}

\begin{lemma}
 The following considerations are true:
 \begin{enumerate}
  \item if u harmonic in $\Omega$ with $B_R(\mathbf{p})\subset\subset\Omega$ then
   \begin{equation}
    |u_{x_j}(\mathbf{p})|\leq\frac{n}{R}\max_{\partial B_R(\mathbf{p})}|u|
   \end{equation}
  \item if u harmonic in outer domain $\mathbb{R}^n\backslash B_a $ and bounded $u\leq M$:
   \[(k+1)a\leq |\mathbf{p}| \leq (k+2)a \leq 2ka \]
   \begin{equation}
    |u_{x_j}(\mathbf{p})|\leq\frac{n}{k\,a}\max_{\partial B_{k\,a}(\mathbf{p})}|u|\leq\frac{2n}{r}M = \frac{c}{r}
   \end{equation}

 \end{enumerate}
\end{lemma}
\fi

