\section{The physics of the problem}
Let $\vec{\mathcal{E}}(x)$ and $\vec{\mathcal{H}}(x)$ denote time-harmonic electric and magnetic fields
\begin{equation}
 \vec{\mathcal{E}}(x) = \vec{E}(x)e^{-\i\omega t}\quad\vec{\mathcal{H}}(x)=\vec{H}(x)e^{-\i\omega t}
\end{equation}
which satisfy Maxwell's equations
\begin{align}
&\divergence \vec{\mathcal{D}} = \rho &&\curl\vec{\mathcal{E}} = -\frac{\partial\vec{\mathcal{B}}}{\partial t} \\
&\divergence \vec{\mathcal{B}} = 0 &&\curl\vec{\mathcal{H}} = \vec{j} +\frac{\partial\vec{\mathcal{D}}}{\partial t}
\end{align}
and there holds the following constitutive relations
\begin{equation}
 \vec{\mathcal{D}} = \epsilon \vec{\mathcal{E}} = \epsilon_0\kappa\vec{\mathcal{E}}\quad \vec{\mathcal{B}} =\mu\vec{\mathcal{H}} = \mu_0\kappa_m\vec{\mathcal{H}}
\end{equation}
and Ohm's law $\vec{j} = \sigma \vec{\mathcal{E}}$
\par
As introduced in \cite{kirsch:book}, \cite{borcea:eit}, Maxwell's equations for harmonic fields, in the space become
\begin{align}
 &\curl\vec{E} = \i\omega\mu\vec{H} \\
 &\curl\vec{H} = \gamma\vec{E} \quad\quad \gamma =\sigma -\i\omega\epsilon
\end{align}
where $\sigma$ is conductivity, $\epsilon$ is electric permittivity, $\mu$ is magnetic permeability
\par
If the length $L$ is a typical scale of $|\Omega|^{1/2}$, and in general $\omega\mu|\gamma| L^2 \ll 1$, the scaled formulation is
\begin{center}
 find $\vec{E}, \vec{H}$ free divergence fields which satisfy
 \begin{align}
  &\curl\vec{E} = 0 \\
  &\curl\vec{H} = \gamma \vec{E}
 \end{align}
\end{center}
If the domain $\Omega$ is simply connected, the first equation is solved by any $\vec{E} = -\nabla u$, and the scalar potential $u$ is the only unknown of the restated second equation
\begin{equation}
 \divergence(\gamma\nabla u) = - \divergence\curl \vec{H} = 0\quad \textup{ in }\Omega
\end{equation}
\begin{definition}
 Some recaps:
 \begin{itemize}
  \item \emph{impedance} $Z$ is the generalized complex valued conductivity, and the generalized Ohm's law is 
  \begin{equation}
   V = IZ
  \end{equation}
  \item \emph{admittance} $Y$ is the reciprocal of impedance
 \end{itemize}
\end{definition}

\begin{assumption}
\label{assumption:admittance}
The \emph{admittance} $\gamma(x)\in\mathbb{C}^{n\times n}$ is defined in terms of the \emph{conductivity} $\sigma(x)\in\mathbb{R}^{n\times n}$ and \emph{permittivity} $\epsilon(x)\in\mathbb{R}^{n \times n}$ with frequency $\omega$
\begin{equation}
 \gamma(x) = \sigma(x) - \text{i}\omega \epsilon(x)
\end{equation}
We reduce to the simple case
\begin{equation}
 \gamma(x) = 1 + (k(x) - 1 ) \cdot \chi_D(x) = 1 + h(x)\cdot\chi_D(x)
\end{equation}
with $h(x) \in C^1(D?;\,\mathbb{C}^{n\times n})$ and
\begin{align}
 &\exists\, \alpha>0\,\forall \zeta \in \mathbb{C}\quad \real(?z\overline{\zeta}\cdot\gamma(x)\zeta)     \geq\,  \alpha|\zeta|^2 \quad x\in \Omega, \text{ for some } z \in \mathbb{C}\\
 &\exists\, \beta>0\,\forall \zeta \in \mathbb{C}\quad \imag(\overline{\zeta}\cdot\gamma(x)\zeta)        \leq\,  -\beta|\zeta|^2 \quad x\in \mathcal{O} \subset D \text{ some open set }
\end{align}
\end{assumption}
\begin{assumption}
In general, for the weak formulation it's sufficient, as in \cite{kirsch:book}
\begin{align}
 & \gamma(x)\in L^\infty(\Omega;\,\mathbb{R} ) \\
 & \exists \gamma_0>0\textup{ such that the coercivity condition holds }\gamma(x)\geq \gamma_0 > 0
\end{align}
or
\begin{align}
 & h(x)\in L^\infty(D;\,\mathbb{R} ) \\
 & \exists h_0>0\textup{ such that the coercivity condition holds }h(x)\geq h_0 > 0
\end{align}
\end{assumption}

