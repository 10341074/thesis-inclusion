\section{Appendix A: Functional analysis}
\begin{theorem}[Parseval]
 Let $X$ be a separable Hilbert space and $\{x_j\}_{j\in\mathbb{N}}$ a complete orthonormal basis, then
 \begin{enumerate}
  \item for any $\{\lambda_j\}_{j\in\mathbb{N}} \in \ell ^2$, the sequence $\sum_{j}\lambda_jx_j$ converges in $X$ (also with a non complete basis)
  \item for any $x \in X$ orthogonal decomposition (Fourier) holds
  \begin{equation}
   x = \sum_{j \in \mathbb{N}} \langle x, x_j\rangle x_j \quad \text{and} \quad \|x\|^2=\sum_{j \in \mathbb{N}} |\langle x, x_j\rangle|^2
  \end{equation}
 \end{enumerate}
\end{theorem}

\begin{theorem}[Spectral Theorem]
 Let $X$ be an Hilbert space and $K:X\to X$ be a compact self-adjoint linear operator, then the spectrum contains only the regular part and the eigenvalues are:
 \begin{enumerate}
  \item a finite set, (0 is eigenvalue with correspondent eigenspace of infinite dimension)
  \item an infinite sequence converging to 0, (0 can be or not an eigenvalue with correspondent eigenspace with finite or infinite dimension)
 \end{enumerate}
 and eigenvectors can be taken such that they constitute a complete orthonormal basis of $X$
 \begin{center}
  $\{(\lambda_j, x_j)\}_{j\in\mathbb{N}}$ is an orthonormal eigensystem
 \end{center}
\end{theorem}

\begin{definition}[Singular Values]
 Let $X$ and $Y$ be Hilbert spaces, and $K:X\to Y$ be a compact linear operator. Then, by spectral theorem, there exists an orthonormal eigensystem $\{(\lambda_j,x_j)\}_{j\in\mathbb{N}}$ with nonnegative eigenvalues $\lambda_j \geq 0$, from the self-adjoint compact nonnegative operator $K^*K$. The square roots $\mu_j = \sqrt{\lambda_j}$ are called singular values of $K$
\end{definition}

\begin{theorem}[Singular Value Decomposition]
Let $K: X\to Y$ be a compact linear operator and $\mu_1\geq\mu_2...>0$ its positive singular values, then there exist two orthogonal bases $\{x_j\}_{j \in \mathbb{N}}\subset X$ and $\{y_j\}_{j \in \mathbb{N}}\subset Y$ such that
\begin{equation}
 Kx_j = \mu_j y_j \quad \text{and}\quad K^* y_j = \mu_j x_j
\end{equation}
The system $\{(\mu_j,x_j,y_j)\}_{j \in \mathbb{N}}$ is called singular system and for any $x \in X$ there holds singular value decomposition (Fourier)
\begin{align}
 & x = x_0 + \sum_{j \in \mathbb{N}} \langle x, x_j \rangle x_j\quad \text{with}\quad x_0 \in \mathcal{K}(K) \\
 & Kx = \sum_{j \in \mathbb{N}} \mu_j \langle x, x_j \rangle y_j
\end{align}
\end{theorem}

\begin{theorem}[Picard]
Let $K: X \to Y$ be a compact linear operator with singular system $\{(\mu_j,x_j,y_j)\}_{j \in \mathbb{N}}$, then the equation
\begin{gather}
 Kx = y \,\,\text{is solvable}\\
 \text{if and only if}\\
 y \in \mathcal{K}(K^*)^\perp\quad\text{and} \quad \sum_{j \in \mathbb{N}}\frac{1}{\mu_j^2}|\langle y,y_j\rangle|^2< \infty
\end{gather}
In this case a possible solution is
\begin{equation}
 x = \sum_{j \in \mathbb{N}}\frac{1}{\mu_j}\langle y,y_j\rangle x_j
\end{equation}
\end{theorem}

