\documentclass[10pt]{beamer}
% \documentclass[handout]{beamer}
\usetheme[secheader]{Boadilla}
%%%%%%%%%%%%%%%%%%%%%%%%%%%%%%%

% \usepackage[english]{babel}
\usepackage[italian, english]{babel}

\usepackage[utf8]{inputenc} % needed for bibtex
\usepackage{fontenc}
\usepackage{mathrsfs}
\usepackage{amsthm}
\usepackage{amsmath}
\usepackage{amssymb}
\usepackage{mathtools}
%\usepackage{nccmath} % mfrac
%\mathbb needs amsfonts or amssymb
\usepackage{amsfonts}
\usepackage{bm} % bold symbols math

% \usepackage[overload]{empheq} % left\{ for align % no crash kile ?
\usepackage{cases}
% \usepackage[usenames,dvipsnames]{xcolor} % before tikz, options for more colors
\definecolor{light-gray}{gray}{0.95}
\definecolor{dark-gray}{gray}{0.65}

% \usepackage[sort, numbers]{natbib}
% \setcitestyle{square}
% \usepackage{bbold}

\usepackage{graphicx}

\usepackage{booktabs}
\usepackage{caption}
\usepackage{subfig}
% \captionsetup[figure]{width=.85\textwidth}
\captionsetup[subfigure]{margin=0.5cm}
\usepackage{tikz}
\usetikzlibrary{matrix}
\usetikzlibrary{positioning}
% set arrows as stealth fighter jets
\tikzset{>=stealth}
% bezier
\usetikzlibrary{decorations.pathreplacing}
\tikzset{%
  show curve controls/.style={
    postaction={
      decoration={
        show path construction,
        curveto code={
          \draw [blue] 
            (\tikzinputsegmentfirst) -- (\tikzinputsegmentsupporta)
            (\tikzinputsegmentlast) -- (\tikzinputsegmentsupportb);
          \fill [red, opacity=0.5] 
            (\tikzinputsegmentsupporta) circle [radius=.5ex]
            (\tikzinputsegmentsupportb) circle [radius=.5ex];
        }
      },
      decorate
}}}
\tikzstyle{mybox} = [draw=gray, fill=light-gray, very thick,
    rectangle, rounded corners, inner sep=10pt, inner ysep=20pt]
\tikzstyle{mytitle} =[fill=gray, text=white]

% plots
\usepackage{pgfplots}

%\usepackage{color}

\usepackage{xcolor}
\definecolor{bookColor}{cmyk}{1 , 1  , 0   , 0}  % 0.90\% of black
%\color{bookColor}

\usepackage{hyperref}
% \hypersetup{pdftex,colorlinks=true,allcolors=blue}

% % \usepackage[hidelinks]{hyperref}
% % \usepackage{xcolor}
% \AtBeginDocument{
% \hypersetup{
% % %     colorlinks=false,
% % %     citebordercolor = {Green},
% % %     filebordercolor = {Blue}
%     linkbordercolor = {Blue}
% % %     linkcolor={red!50!black},
% % %     citecolor={blue!50!black},
% % %     urlcolor={blue!80!black}
% }
% }


\usepackage{hypcap}
\usepackage{etoolbox}

\usepackage{csquotes}
%\usepackage[autostyle, italian=guillemets]{csquotes}

% \usepackage[chapter]{placeins} % floatbarrier

\usepackage[backend=biber, style=alphabetic]{biblatex}
\addbibresource{sources.bib}
\DeclareFieldFormat[article]{title}{\textit{#1}}
%%%%%%%%%%%%%%%%%%%%%%%%%%%%%%%%%%%%%%%%%%%%%%%%%%%%%%%%%%%%%%%%%%%%%%%%%%%%%%%%%%%%%%%%%%%%%%%%%%%%%%%
%%%%%%%%%%%%%%%%%%%%%%%%%%%%%%%%%%%%%%%%%%%%%%%%%%%%%%%%%%%%%%%%%%%%%%%%%%%%%%%%%%%%%%%%%%%%%%%%%%%%%%%
%%%%%%%%%%%%%%%%%%%%%%%%%%%%%%%%%%%%%%%%%%%%%%%%%%%%%%%%%%%%%%%%%%%%%%%%%%%%%%%%%%%%%%%%%%%%%%%%%%%%%%%
%%%%%%%%%%%%%%%%%%%%%%%%%%%%%%%%%%%%%%%%%%%%%%%%%%%%%%%%%%%%%%%%%%%%%%%%%%%%%%%%%%%%%%%%%%%%%%%%%%%%%%%
\usepackage{smartdiagram}
\usesmartdiagramlibrary{additions}
%%%%%%%%%%%%%%%%%%%%%%%%%%%%%%%%%%%%%%%%%%%%%%%%%%%%%%%%%%%%%%%%%%%%%%%%%%%%%%%%%%%%%%%%%%%%%%%%%%%%%%%
%%%%%%%%%%%%%%%%%%%%%%%%%%%%%%%%%%%%%%%%%%%%%%%%%%%%%%%%%%%%%%%%%%%%%%%%%%%%%%%%%%%%%%%%%%%%%%%%%%%%%%%
%%%%%%%%%%%%%%%%%%%%%%%%%%%%%%%%%%%%%%%%%%%%%%%%%%%%%%%%%%%%%%%%%%%%%%%%%%%%%%%%%%%%%%%%%%%%%%%%%%%%%%%
%%%%%%%%%%%%%%%%%%%%%%%%%%%%%%%%%%%%%%%%%%%%%%%%%%%%%%%%%%%%%%%%%%%%%%%%%%%%%%%%%%%%%%%%%%%%%%%%%%%%%%%
%%%%%%%%%%%%%%%%%%%%%%%%%%%%%%%%%%%%%%%%%%%%%%%%%%%%%%%%%%%%%%%%%%%%%%%%%%%%%%%%%%%%%%%%%%%%%%%%%%%%%%%
%%%%%%%%%%%%%%%%%%%%%%%%%%%%%%%%%%%%%%%%%%%%%%%%%%%%%%%%%%%%%%%%%%%%%%%%%%%%%%%%%%%%%%%%%%%%%%%%%%%%%%%
%%%%%%%%%%%%%%%%%%%%%%%%%%%%%%%%%%%%%%%%%%%%%%%%%%%%%%%%%%%%%%%%%%%%%%%%%%%%%%%%%%%%%%%%%%%%%%%%%%%%%%%
%%%%%%%%%%%%%%%%%%%%%%%%%%%%%%%%%%%%%%%%%%%%%%%%%%%%%%%%%%%%%%%%%%%%%%%%%%%%%%%%%%%%%%%%%%%%%%%%%%%%%%%
\setbeamerfont{block title}{size=\normalsize}
% \theoremstyle{definition}
% \newtheorem{definition}[subsection]{Definition}
% \theoremstyle{plain}
% \newtheorem{theorem}[subsection]{Theorem}
% \theoremstyle{plain}
% \newtheorem{corollary}[subsection]{Corollary}
% \theoremstyle{plain}
\newtheorem{proposition}[subsection]{Proposition}
\theoremstyle{plain}
\newtheorem{remark}[subsection]{Remark}
\theoremstyle{plain}
% \newtheorem{lemma}[subsection]{Lemma}
% \theoremstyle{plain}
% \newtheorem{example}[subsection]{Example}
% 
% \theoremstyle{plain}
% \newtheorem{assumption}[subsection]{Assumption}
% \theoremstyle{plain}
% \newtheorem{problem}[subsection]{Problem}

\DeclareMathOperator{\divergence}{div}
\DeclareMathOperator{\curl}{curl}
\DeclareMathOperator{\real}{Re}
\DeclareMathOperator{\imag}{Im}

\renewcommand{\i}{\textup{i}}
\let\phi\varphi
\let\epsilon\varepsilon
%%%%%%%%%%%%%%%%%%%%%%%%%%%%%%
\usepackage{amssymb,tikz}

\newcommand{\mysetminusD}{\hbox{\tikz{\draw[line width=0.6pt,line cap=round] (3pt,0) -- (0,6pt);}}}
\newcommand{\mysetminusT}{\mysetminusD}
\newcommand{\mysetminusS}{\hbox{\tikz{\draw[line width=0.45pt,line cap=round] (2pt,0) -- (0,4pt);}}}
\newcommand{\mysetminusSS}{\hbox{\tikz{\draw[line width=0.4pt,line cap=round] (1.5pt,0) -- (0,3pt);}}}

\newcommand{\mysetminus}{\mathbin{\mathchoice{\mysetminusD}{\mysetminusT}{\mysetminusS}{\mysetminusSS}}}
%%%%%%%%%%%%%%%%%%%%%%%%%%%%%%%%%%


\usepackage{environ}
\NewEnviron{mybox}{%
\begin{center}
\colorbox{light-gray}{\color{black}\parbox{\textwidth}{%
% \fcolorbox{gray}{light-gray}{
\BODY
}}
\end{center}
}

\usepackage{fancyhdr}
\newcommand{\fncyblank}{\fancyhf{}}
%%%%%%%%%%%%%%%%%%%%%%%%%%%%%%%%%%%%%%%%%%%%%%%%%%%%%%%%%%%%%%%%%%%%%%%%%%%%%%%%%%%%%%%%%%%%
% abstract e sommario
% \newenvironment{abstract} %
% {\cleardoublepage
% \fncyblank\null\vfill\begin{center} %
% \bfseries\abstractname
% \end{center}} %
% {\vfill\null}
% \newenvironment{abstractone} %
% {\clearpage
% \fncyblank\null\vfill\begin{center} %
% \bfseries\abstractname
% \end{center}} %
% {\vfill\null}


%%%%%%%%%%%%%%%%%%%%%%%%%%%%%%%%%%%%%%%%%%%%%%%%%%%%%%%%%%%%%%%%%%%%%%%%%%%%%%%%%%%%%%%%%%%%%
\title{Numerical Reconstruction of Inclusions in Electrical Conductors}
\author{Giacomo Milan}
\date{3 Ottobre 2017}

%%%%%%%%%%%%%%%%%%%%%%%%%%%%%%%%%%%%%%%%%%%%%%%%%%%%%%%%%%%%%%%%%%%%%%%%%%%%%%%%%%%%%%%%%%%%%%%%%%%%%%%%%%%%%
%%%%%%%%%%%%%%%%%%%%%%%%%%%%%%%%%%%%%%%%%%%%%%%%%%%%%%%%%%%%%%%%%%%%%%%%%%%%%%%%%%%%%%%%%%%%%%%%%%%%%%%%%%%%%
%%%%%%%%%%%%%%%%%%%%%%%%%%%%%%%%%%%%%%%%%%%%%%%%%%%%%%%%%%%%%%%%%%%%%%%%%%%%%%%%%%%%%%%%%%%%%%%%%%%%%%%%%%%%%
%%%%%%%%%%%%%%%%%%%%%%%%%%%%%%%%%%%%%%%%%%%%%%%%%%%%%%%%%%%%%%%%%%%%%%%%%%%%%%%%%%%%%%%%%%%%%%%%%%%%%%%%%%%%%
%%%%%%%%%%%%%%%%%%%%%%%%%%%%%%%%%%%%%%%%%%%%%%%%%%%%%%%%%%%%%%%%%%%%%%%%%%%%%%%%%%%%%%%%%%%%%%%%%%%%%%%%%%%%%
\begin{document}
%%%%%%%%%%%%%%%%%
\begin{frame}
 \maketitle
\end{frame}
%%%%%%%%%%%%%%%%%
\begin{frame}
 \frametitle{}
 \begin{center}
% \smartdiagram[descriptive diagram]{
% {Set up,The set up operation consist of..},
% }
\smartdiagramset{
additions={
additional item border color=red,
additional item width = 8cm,
}}
\smartdiagramadd[flow diagram:horizontal]{%
PGF,Ti\textit{k}Z,Smartdiagram%
}{%
below of module1/Low Level, above of module1/High level%
}
\smartdiagramconnect{<->}{module1/module2}
\end{center}
\end{frame}
%%%%%%%%%%%%%%%%%
%  \only<1>{GA}\only<2>{MOGA}\only<3>{pMOGA}
\begin{frame}
 \frametitle{Neumann--to--Dirichlet maps}
  Let $v\in H^1_0(\Omega)$ be the unique solution of the {\color{blue}Laplace problem}
 \begin{columns}[T]
  \begin{column}{0.5\textwidth}
    \begin{center}
    \begin{tikzpicture}
    \draw [fill=light-gray, draw=gray] (0,0.2) ellipse (2cm and 1.5cm);
%     \node at (-2.2,0.7){${\partial \Omega}$};
    \node at (-1.2,0.1){$\Omega, 1$};
    \end{tikzpicture}
    \end{center}
  \end{column}
  \hspace{-2cm}
  \begin{column}{0.5\textwidth}
  \vspace{0.6cm}
  \begin{center}
 \begin{equation}
 \label{eq:NtoD-inclusion}
  \left\{
  \begin{aligned}
  \divergence(\nabla v) &= 0, \quad\text{in}\,\Omega, \\
            \partial_\nu v &= f, \quad\text{on}\,\partial \Omega,
  \end{aligned}
  \right.
 \end{equation}
 \end{center}
 \end{column}
 \end{columns}
 \vspace{0.6cm}
 and define
   \begin{align}
   &{N_0}: H^{-1/2}_0(\partial \Omega) \to H^{1/2}_0(\partial\Omega) && {N_0} f = v|_{\partial\Omega},
   \end{align}
\end{frame}
%%%%%%%%%%%%%%%%%
\begin{frame}
 \frametitle{Neumann--to--Dirichlet maps}
%   Let \only<1>{$v\in H^1_0(\Omega)$}\only<2>{$u\in H^1_0(\Omega)$} be the unique solution of the
%   \only<1>{{\color{blue}Laplace problem}}\only<2>{{\color{blue} inclusion problem} with $D\subset \Omega$}
  Let $u\in H^1_0(\Omega)$ be the unique solution of the
  {\color{blue} inclusion problem} with $D\subset \Omega$
 \begin{columns}[T]
  \begin{column}{0.5\textwidth}
  \begin{center}
    \begin{tikzpicture}
    \draw [fill=light-gray, draw=gray] (0,0.2) ellipse (2cm and 1.5cm);
    \filldraw [fill=dark-gray, draw=gray] (0, -0.1) 
     .. controls ++(165:-0.2) and ++(90:-0.5) .. ( 1, 0.1)
     .. controls ++(90:0.5) and ++(1:0.5) .. (0.2, 0.5)
     .. controls ++(1:-0.5) and ++(60:0.7) .. (-0.6, 0.2)
     .. controls ++(60:-0.7) and ++(165:0.2) .. ( 0, -0.1);
%     \node at (-2.2,0.7){${\partial \Omega}$};
%     \node at (-0.6,0.7){${\partial D}$};
    \node at (-1.2,0.1){$\Omega, 1$};
    \node at (0.1,0.1){$D, k$};
    \end{tikzpicture} 
    \end{center}
  \end{column}
 \hspace{-2cm}
 \begin{column}{0.5\textwidth}
 \vspace{0.6cm}
 \begin{center}
 \begin{equation}
 \label{eq:NtoD-inclusion}
  \left\{
  \begin{aligned}
  \divergence(\gamma\nabla u) &= 0, \quad\text{in}\,\Omega, \\
            \partial_\gamma u &= f, \quad\text{on}\,\partial \Omega,
  \end{aligned}
  \right.
 \end{equation}
 \end{center}
  \end{column}
 \end{columns}
 \vspace{0.6cm}
 with $\gamma(x) = 1\chi_{\Omega\backslash\overline{D}} + k\,\chi_D$, and define
\begin{align}
 &{N_D}: H^{-1/2}_0(\partial \Omega) \to H^{1/2}_0(\partial\Omega) && {N_D}f = u|_{\partial\Omega}.
 \end{align}
\end{frame}
%%%%%%%%%%%%%%%%%
\begin{frame}
 \frametitle{Linear sampling method: the map}
%  In the inverse inclusion problem, a crucial role is played by the 
We denote by ${N_r}:H_0^{\,-1/2}(\partial \Omega)\to H_0^{1/2}(\partial \Omega)$
the {\color{blue} relative} Neumann--to--Dirichlet map
% (see \cite{somersalo:preprint})
\begin{equation}
 {N_r} \coloneqq {N_0} - {N_D}.
\end{equation}
  \begin{proposition}
  The operator ${N_0} - {N_D}:H_0^{\,-1/2}(\partial \Omega)\to H_0^{1/2}(\partial \Omega)$ 
  is injective, with dense image, self-adjoint, compact and positive.
 \end{proposition}
%  This is proved in \cite{kirsch:book}, but we present the proof for injectivity contained in \cite{somersalo:preprint}, to highlight the link with ITP.
\end{frame}
%%%%%%%%%%%%%%%%%
\begin{frame}
 \frametitle{Linear sampling method: the right term}
  As right term we consider the potential generated by a dipole
 \begin{equation}
 \left\{
 \begin{aligned}
   -\Delta \psi(x,z) &= \nabla\delta_{z} \cdot \vec{d}, && \textup{ in }\Omega ,\\
   \partial_\nu\psi(x,z) &= 0, &&\textup{ on }\partial \Omega,
 \end{aligned}
 \right.
 \end{equation}
 with unit vector $|\vec{d}\,|=1$.
 \par
 We use the notation
 \begin{equation}
 \vec{\Psi}(x,z)\coloneqq\nabla_x\Phi(x,z) 
 \end{equation}
 for the derivative of the fundamental solution.
 \begin{definition}
 \label{def:lsm-psi}
 We denote
 \begin{equation}
 \psi_{0z} \coloneqq \vec{\Psi}(x,z)\cdot\vec{d} - m_z - N_0\bigl(\partial_\nu \vec{\Psi}(x,z) \cdot \vec{d}\bigr),
  \end{equation}
%   and
%   \begin{equation}
%   \psi_{z} \coloneqq \vec{\Psi}(x,z)\cdot\vec{d} - m_z - M_D\bigl(\partial_\nu \vec{\Psi}(x,z) \cdot \vec{d}\bigr),
%   \end{equation}
  with $m_z$ the mean of $\vec{\Psi_z}\cdot\vec{d}$ on $\partial \Omega$.
 \end{definition}
\end{frame}
%%%%%%%%%%%%%%%%%
\begin{frame}
 \only<1>{
 \frametitle{Single layer potential}}
\begin{columns}[T]
  \begin{column}{0.50\textwidth}
  \includegraphics[width=1.1\textwidth]{fig/Spng}
  \end{column}
  \begin{column}{0.50\textwidth}
  We define
   \begin{equation*}
    \mathcal{S}(\partial D,\psi)(x)\coloneqq \int_{\partial D} \Phi(x, y)\psi(y)\, dy,
%   \quad x\in\mathbb{R}^m \backslash\partial D, 
    \label{eq:definition-single-layer}
   \end{equation*}
  with a jump on $\partial\Omega$ in the first derivative
  \begin{align*}
%    \mathcal{S}^\pm(z) &\coloneqq\lim_{h\to 0^\pm}\mathcal{S}(z+h\nu(z)) = S\psi(z)=\int_{\partial D}\psi(y)\Phi(z,y)\,dy \quad z\in\partial D, \label{eq:single-pm-0}\\
   \partial_\nu\mathcal{S}^\pm(z) =  K'\psi(z) \,\mp\,\dfrac{1}{2}\psi(z) \quad z\in\partial D.\label{eq:single-pm-1}
  \end{align*}
  \end{column}
 \end{columns}
 \end{frame}
%%%%%%%%%%%%%%%%%
\begin{frame}
 \frametitle{Double layer potential}
\begin{columns}[T]
  \begin{column}{0.50\textwidth}
  \vspace{-0.2cm}
  \includegraphics[width=1.1\textwidth]{fig/Dpng}
  \end{column}
  \begin{column}{0.50\textwidth}
   We define
   \begin{equation*}
    \mathcal{D}(\partial D,\psi)(x)\coloneqq \int_{\partial D} \partial_{\nu(y)}\Phi(x, y)\psi(y)\, dy,
%   \quad x\in\mathbb{R}^m \backslash\partial D. 
    \label{eq:definition-double-layer}
   \end{equation*}
   with a jump on $\partial\Omega$
  \begin{align*}
   \mathcal{D}^\pm(z) &= K\psi(z) \pm\dfrac{1}{2}\psi(z)\quad z\in\partial D. \label{eq:double-pm-0}
%    \\
%    \partial_\nu\mathcal{D}^+(z) &= \partial_\nu\mathcal{D}^-(z) \quad z\in\partial D. \label{eq:double-pm-1}
  \end{align*}

  \end{column}
 \end{columns}
 \end{frame}
%%%%%%%%%%%%%%%%%
\begin{frame}
 \frametitle{Approximating sequence for $z\in D$}
 We split the main result in two propositions.
\begin{proposition}[Constructive Part]
\label{prop:lsm-constructive}
Fix $z \in D$. Then for any $\epsilon > 0$ there exists an approximating harmonic layer $\mathcal{S}(\partial B, \omega^\epsilon_z)$ with density $\omega^\epsilon_z\in L^2(\partial B)$  such that
\begin{equation}
 \|({N_D} - {N_0})\partial_\nu\mathcal{S}(\partial B, \omega^\epsilon_z) - \psi_{0z}\|_{H^{1/2}(\partial\Omega)} < \epsilon,
\end{equation}
furthermore, when $z$ approaches the boundary $\partial D$, $\|\omega^\epsilon_z\|_{L^2(\partial B)}\to + \infty$.
\end{proposition}
\begin{remark}
 For fixed $z\in D$, there exist a sequence 
 $v^{\epsilon_n}_z\coloneqq\mathcal{S}(\partial B, \omega^{\epsilon_n}_z)$ 
 such that it's approximating as above for $\epsilon_n \to 0$,
%  that is
% \begin{equation}
%  \|({N_D} - {N_0})\partial_\nu v^{\epsilon_n}_z - \psi_{0z}\|_{H^{1/2}(\partial\Omega)} < \epsilon_n, \quad \textup{ for }\epsilon_n\to 0,
% \end{equation}
 and it is converging to some function $v_z\in H^1(D)$
\begin{equation}
 \|v^{\epsilon_n}_z - v_z\|_{H^1(D)} \to 0.
\end{equation}
 Obviously, $\{v^{\epsilon_n}_z\}_{n\in\mathbb{N}}$ is bounded in the same norm $H^1(D)$.
\end{remark}

\end{frame}
%%%%%%%%%%%%%%%%%
\begin{frame}
 \frametitle{Approximating sequence for $z\notin D$}
 \begin{proposition}[Counterpart]
\label{prop:lsm-counterpart}
Fix $z \in \Omega\backslash\overline{D}$. Then for any $\delta>0$ and $\epsilon > 0$ there exists an harmonic layer $\mathcal{S}(\partial B, \omega^{\delta, \epsilon}_z)$ with density $\omega^{\delta, \epsilon}_z\in L^2(\partial B)$ such that
\begin{equation}
 \|({N_D} - {N_0})\partial_\nu\mathcal{S}(\partial B, \omega^{\delta,\epsilon}_z) - \psi_{0z}\|_{H^{1/2}(\partial\Omega)} < \delta + \epsilon,
\end{equation}
and $\|\omega^{\delta, \epsilon}_z\|_{L^2(\partial B)}\to + \infty$ as $\delta\to 0$.
\end{proposition}
In the next section, we will compare the plots of the norm of the computed $f$, according to Propositions
\ref{prop:lsm-constructive} and \ref{prop:lsm-counterpart}. To take in account the norm of right term 
$\psi_{0z}$, we will plot the inverse of the ratio of the two norms, namely
\begin{equation}
 r_n(z)\coloneqq\frac{\|\psi_{0zn}\|_{L^2(\partial \Omega)}}{\|f_n\|_{L^2(\partial \Omega)}},
\end{equation}
such that $\|r_n(z)\|\to 0$ as $\|f_n\|\to \infty$.

\end{frame}
%%%%%%%%%%%%%%%%%
\begin{frame}[noframenumbering]
\begin{center}
\Large
 2. Reciprocity gap method
\end{center}
\end{frame}
%%%%%%%%%%%%%%%%
\begin{frame}
 \frametitle{Reciprocity gap method}
 \begin{columns}[T]
 \begin{column}{0.50\textwidth}
%   \begin{tikzpicture}
%   \draw [dashed] (2,2) ellipse (3cm and 2cm);
%   \draw [dashed] (2,2) ellipse (1.5cm and 1cm);
%   \draw (2,2) circle (0.5cm);
%   %\draw (2,2) rectangle (1cm and 3cm);
%   \node at (-0.5,2){$B$};
%   \node at (1,2){$\Omega$};
%   \node at (2,2){$D,k$};
%  \end{tikzpicture}
    \begin{center}
    \begin{tikzpicture}
    \draw [draw=gray] (0,0.2) ellipse (2cm and 1.5cm);
    \draw [fill=light-gray, draw=gray] (0,0.2) ellipse (1.5cm and 1cm);
    \filldraw [fill=dark-gray, draw=gray] (0, -0.1) 
     .. controls ++(165:-0.2) and ++(90:-0.5) .. ( 1, 0.1)
     .. controls ++(90:0.5) and ++(1:0.5) .. (0.2, 0.5)
     .. controls ++(1:-0.5) and ++(60:0.7) .. (-0.6, 0.2)
     .. controls ++(60:-0.7) and ++(165:0.2) .. ( 0, -0.1);
%     \node at (-2.2,0.7){${\partial \Omega}$};
%     \node at (-0.6,0.7){${\partial D}$};
    \node at (-1.2,0.1){$\Omega, 1$};
    \node at (0.1,0.1){$D, k$};
    \end{tikzpicture} 
    \end{center}
 \end{column}
 \begin{column}{0.50\textwidth}
 \vspace{0.8cm}
  \begin{itemize}
   \item $\partial B$ sources boundary
   \item $\partial \Omega$ observation boundary
   \item data
   \begin{equation*}
    u\in\mathcal{U} \to u|_{\partial\Omega},\, \partial_\nu u|_{\partial\Omega}
   \end{equation*}

  \end{itemize}
 \end{column}
 \end{columns}
%  In concrete applications, the function $u$ is not computed, though it's observed on the 
% surface $\partial\Omega$ and the measures $u|_{\partial \Omega}$, 
% $\partial_\nu u|_{\partial \Omega}$ are considered for the definition of the reciprocity gap operator. Dirichlet data $f$ are the values of the generic potential which we are able to generate from the exterior: it can be a plane wave in the scattering case, or the potential generated by a point source $\delta_{x_0}$. In the latter case we are referring to the fundamental solution, and by uniqueness, the same $u$ is equal to $u(x)=\Phi_D(x,x_0)$ for $x\in \Omega$.
% In the sequel $B$ is the background medium which contains $\Omega\subset B$.

% \begin{definition}
% \label{def:setU}
% We will denote by $\mathcal{U}$ the set of functions
% \begin{equation}
%  \mathcal{U}\coloneqq\bigl\{\Phi_D(x, x_0): x_0\in \partial B\bigr\},
% \end{equation}
% where $\Phi_D$ is the fundamental solution, which solves the inclusion problem.
% \end{definition}
\vspace{0.8cm}
We denote by (where $\Phi_D$ is the fundamental solution of the inclusion problem)
\begin{itemize}
 \item $\mathcal{U}$ the set of functions
$
 \mathcal{U}\coloneqq\bigl\{\Phi_D(x, x_0): x_0\in \partial B\bigr\},
$
 \item $\mathcal{V}(\overline{B})$ the set of harmonic functions, continuously defined on $\overline{B}$,
 \item $\mathcal{V}_0(\overline{B})$ the subspace 
$
  \mathcal{V}_0(\overline{B})\coloneqq\Bigl\{v\in\mathcal{V}(\overline{B}):\int_{\partial B} v = 0\Bigr\}.
$
\end{itemize}
\end{frame}
%%%%%%%%%%%%%%%%%
% \begin{frame}
%  \frametitle{Reciprocity gap method: definitions}
%  \begin{definition}
% \label{def:setV}
% We denote
% % Let $u\in\mathcal{U}$, the set in Definition \ref{def:setU}. In our case $u(x,x_0)=\Phi_D(x,x_0)$, with $x_0 \in \partial B$, then
% \begin{itemize}
%  \item by $\mathcal{V}(\overline{B})$ the set of harmonic functions, continuously defined on $\overline{B}$,
%  \item by $\mathcal{V}_0(\overline{B})$ the subspace of vanishing mean functions, that is
%  \begin{equation}
%   \mathcal{V}_0(\overline{B})\coloneqq\Bigl\{v\in\mathcal{V}(\overline{B}):\int_{\partial B} v = 0\Bigr\},
% %   =\mathcal{V}(\overline{B})/\textup{span}\bigl(\{1\}\bigr),
%  \end{equation}
% \end{itemize}
% \end{definition}
% \begin{definition}
% We denote by $\textup{R} : \mathcal{V}(\overline{B})\to L^2(\partial B)$ the {\color{blue}reciprocity gap} operator defined by
% \begin{equation}
%  \textup{R}(v)(x_0)\coloneqq 
% %  \mathcal{R}_{\partial\Omega}\bigl(u(\cdot,x_0),v(\cdot)\bigr)\coloneqq \int_{\partial \Omega}\bigl(u(y,x_0)v_\nu (y) - u_\nu(y,x_0)v(y)\bigr)dy.
%  \mathcal{R}_{\partial\Omega}\bigl(u(\cdot,x_0),v(\cdot)\bigr)
%  \coloneqq \int_{\partial \Omega}\bigl(uv_\nu - u_\nu v \bigr)dy.
% \end{equation}
% \end{definition}
% \end{frame}
%%%%%%%%%%%%%%%%%
\begin{frame}
 \frametitle{Reciprocity gap method: the operator R}
 We denote by $\textup{R} : \mathcal{V}(\overline{B})\to L^2(\partial B)$ the {\color{blue}reciprocity gap operator}
 defined by
\begin{equation}
 \textup{R}(v)(x_0)\coloneqq 
%  \mathcal{R}_{\partial\Omega}\bigl(u(\cdot,x_0),v(\cdot)\bigr)\coloneqq \int_{\partial \Omega}\bigl(u(y,x_0)v_\nu (y) - u_\nu(y,x_0)v(y)\bigr)dy.
 \mathcal{R}_{\partial\Omega}\bigl(u(\cdot,x_0),v(\cdot)\bigr)
 \coloneqq \int_{\partial \Omega}\bigl(uv_\nu - u_\nu v \bigr)dy.
\end{equation}

 \begin{proposition}
 The operator $\textup{R} : \mathcal{V}_0(\overline{B})\to L^2(\partial B)$
 \begin{itemize}
  \item[-] is injective,
  \item[-] its range has codimension one: $\mathcal{R}(R)^\perp=\textup{span}(\{\alpha_c\})$,
 \end{itemize}
 where $\alpha_c$ is the solution of 
 $
  \partial_\nu \mathcal{S}^-(\partial B,\alpha_c) = 0,\textup{ on }\partial B.
 $
 \end{proposition}
\end{frame}
%%%%%%%%%%%%%%%%%
\begin{frame}
 \frametitle{Reciprocity Gap Approximation Theorem}
 Finally we can state the key theorem which provides a binary criterion to discriminate the inclusion from the background.
\begin{theorem}[Reciprocity Gap Approximation Theorem]
 \label{theo:approximation-rg} 
 Let $D\subset\Omega\subset B$  be an inclusion satisfying Assumption \ref{assumption:connected}, let $\{(u|_{\partial \Omega},\partial_\nu u|_{\partial \Omega})\}$ 
 be a set of measured data with $u\in\mathcal{U}$ as in Definition \ref{def:setU} and 
 let consider the class $\mathcal{V}(\overline{B})$ of harmonic test functions 
 in Definition \ref{def:setV}. Then
%  At the right hand side, we consider the directional derivative along some unit vector $\vec{d}$ of the fundamental solution, defined above in \ref{def:fund-sol-deriv-Psi}. Then
 \begin{enumerate}
  \item if $z \in D$ then there exists a sequence $\{v_n\} \subset \mathcal{V}(\overline{B})$ such that
   \begin{equation}
     \lim_{n\to\infty}\mathcal{R}(u,v_n) = \mathcal{R}(u,\Psi_z)\quad\forall u\in\mathcal{U},\label{eq:rg-lim-constructive}
   \end{equation}
   where $\Psi_z$ as in Definition \ref{def:fund-sol-deriv-Psi}, and $v_n|_{\partial D}\to g$ in $L^2(\partial D)$ and consequently  it's bounded in the same norm $\|v_n\|_{L^2(\partial D)}$;
  \item if $z \in \Omega \backslash D$ then any sequence $\{v_n\} \subset \mathcal{V}(\overline{B})$ such that
   \begin{equation}
     \lim_{n\to\infty}\mathcal{R}(u,v_n) = \mathcal{R}(u,\Psi_z)\quad\forall u\in\mathcal{U}\label{eq:rg-lim-counterpart}
   \end{equation}
   is unbounded $\|v_n\|_{L^2(\partial D)}\to\infty$ in $L^2(\partial D)$.
 \end{enumerate}
\end{theorem}
\end{frame}
%%%%%%%%%%%%%%%%%
\begin{frame}
 \frametitle{Link}
 We have seen that the reciprocity gap method is formulated as an integral equation 
of the form
\begin{equation}
 \mathcal{R}(u, v) = \mathcal{R}(u, \vec{\Psi}_z\cdot\vec{d})\quad \forall u \in \mathcal{U},
\end{equation}
whereas the linear sampling method deals with the following equation
\begin{equation}
\label{eq:eq-lsm-link}
 ({N_D} - {N_0})f = \psi_{0z} \quad \text{ on }\partial \Omega.
\end{equation}
The number of players suggest to compute the duality of \eqref{eq:eq-lsm-link}, with a generic 
${g \in \bigl(H^{1/2}_0(\partial \Omega)\bigr)^* = H^{\,-1/2}_0(\partial \Omega)}$, by self-adjoint properties (denoting $u,v$ solutions of 
\eqref{eq:NtoD-inclusion}), \eqref{eq:NtoD-laplace}, we have
\begin{align}
 \label{eq:link-duality-left}
 \langle({N_D} - {N_0})f,g\rangle & =  \langle f,{N_D} g\rangle - \langle {N_0} f,g\rangle \\
                                        & = \langle \partial_\nu v,u\rangle - \langle v,\partial_\nu u\rangle \\
                                        & = \mathcal{R}(u,v).
\end{align}
\begin{proposition}
 The \textit{reciprocity gap functional method} correspond to the weak formulation of the \textit{linear sampling method} with a modified right term. Consequently, provided the choice of $\psi_z$ as right side in \eqref{eq:eq-lsm-link}
 instead of $\psi_{0z}$, the two methods are equivalent.
\end{proposition}
\end{frame}
%%%%%%%%%%%%%%%%%
\begin{frame}[noframenumbering]
 \begin{center}
 \Large
 3. Factorization method 
 \end{center}
\end{frame}
%%%%%%%%%%%%%%%%%
\begin{frame}
 \frametitle{Factorization method}
 The aim is to factorize the relative Neumann--to--Dirichlet map ${N_0} - {N_D}$, in the form
\begin{equation}
 {N_r}\coloneqq{N_0} - {N_D} = A^*TA,
\end{equation}
where
% $A,T$ are operators suitably chosen and 
$A^*$ is the adjoint operator of A. 
The choice of the operators is not unique, for instance
\begin{columns}[T]
\begin{column}{0.5\textwidth}
\vspace{0.8cm}
\begin{itemize}
 \item 
%  $A:L^2_0(\partial \Omega) \to L^2(D;\,\mathbb{R}^2)$ defined as 
 $A\,f = \nabla v|_D,$
 \item 
%  $T:L^2(D;\,\mathbb{R}^2)\to L^2(D;\,\mathbb{R}^2)$ as 
 $TA\,f = 
%  (k-1) (\nabla v - \nabla w) =
 (k-1) \nabla u.$
\end{itemize}\end{column}
\begin{column}{0.5\textwidth}
\begin{tikzpicture}
    % set up the nodes
    \node (n11) at (0,0) {$L^2_0(\partial\Omega)$};
    \node[right=of n11] (n12) {$L^2(D;\,\mathbb{R}^2)$};
    \node[below=of n12] (n22) {$L^2(D;\,\mathbb{R}^2)$};
    \node[below=of n11] (n21) {$L^2_0(\partial\Omega)$};
    \node (n112) at ([xshift=0.1]n11) {$ $};
    \node (n212) at ([xshift=0.1]n21) {$ $};
    % draw arrows and text between them
    \draw[->] (n11) to node [midway,right] {${N_r}$} (n21);
    \draw[->] (n11) to node [midway,above] {$A$} (n12);
    \draw[->] (n12) to node [midway,right] {$T$} (n22);
    \draw[->] (n22) to node [midway,above] {$A^*$} (n21);
%     \draw[->] (n21.south) [out=-60, in=-120]to node [midway,below] {$R_\alpha$} (n22.south);
%     \draw[->] (n21.west) [out=170, in=-170]to node [midway,left] {$\tilde{R}_\alpha$} (n11.west);
\end{tikzpicture}
\end{column}
\end{columns}

\end{frame}
%%%%%%%%%%%%%%%%%
\begin{frame}
 \frametitle{Range characterization}
\begin{theorem}
 There holds the equivalence 
 \begin{equation}
  z \in D \,\Longleftrightarrow \,\psi_{0z} \in \mathcal{R}(A^*).
 \end{equation}
\end{theorem}

Both the following factorizations hold
\begin{align*}
\label{eq:factorization-two}
 {N_0} - {N_D}&=({N_0} - {N_D})^{1/2}\,I\,({N_0} - {N_D})^{1/2}\\
 &= A^*TA.
\end{align*}
\vspace{-0.8cm}
 \begin{theorem}
 The range of two operators is the same, that is
 \begin{equation}
  \mathcal{R}(A^*) = \mathcal{R}(({N_0} - {N_D})^{1/2}).
 \end{equation}
Therefore an equivalent criterion can be stated as
 \begin{equation}
  z \in D \,\Longleftrightarrow \,\psi_{0z} \in \mathcal{R}(({N_0} - {N_D})^{1/2}).
 \end{equation}
\end{theorem}
\end{frame}
%%%%%%%%%%%%%%%%%
\begin{frame}
 \frametitle{Range criterion}
We compute the square root of a
compact, self-adjoint and positive operator 
from its spectral decomposition
\begin{align}
({N_0} - {N_D}) f &= \sum_{j\in \mathbb{N}}^\infty \lambda_j(f,\psi_j)_{L^2(\partial\Omega)} \psi_j ,\\
({N_0} - {N_D})^{1/2} f &= \sum_{j\in \mathbb{N}}^\infty \sqrt{\lambda_j}(f,\psi_j)_{L^2(\partial\Omega)} \psi_j.
\end{align}
By Picard's theorem
\begin{equation}
   \psi_{0z} \in \mathcal{R}(({N_0} - {N_D})^{1/2})
   \,\Longleftrightarrow \,
   \sum_{j\in \mathbb{N}}^\infty \frac{(f,\psi_j)^2}{\lambda_j}<\infty. 
\end{equation}
\end{frame}
%%%%%%%%%%%%%%%%%%%%
\begin{frame}
\frametitle{Range criterion}
%  a test to establish the convergence of the previous sum, with finitely many terms, after discretization. 
In most of geometries, the decay of singular values $\bigl\{\lambda_j\bigr\}_j$ is typically exponential
% Let's assume an exponential behavior for the components $(\psi_{0z},\psi_j )^2$, parametrized as
\begin{align}
 \lambda_j &\sim a_j (r_j) ^j,  & \log\lambda_j &\sim c_j + j\log r_j,\label{eq:straightline-eig}\\
 (\psi_{0z}, \psi_j)^2 &\sim A_j(R_j)^j, & \log(\psi_{0z}, \psi_j)^2 &\sim C_j + j\log R_j\label{eq:straightline-data},
\end{align}
then the range criterion leads to comparing the slopes of the straight lines interpolating 
eigenvalues \eqref{eq:straightline-eig} and computed data \eqref{eq:straightline-data}
\begin{equation}
 z\in D \,\Longleftrightarrow\, \dfrac{R_j}{r_j} < 1 \,\Longleftrightarrow\, \log R_j < \log r_j.
\end{equation}
\end{frame}
%%%%%%%%%%%%%%%%%
\begin{frame}
 \frametitle{}
\end{frame}
%%%%%%%%%%%%%%%%%
\begin{frame}
 \frametitle{}
\end{frame}


\end{document}

 \begin{equation}
  (({N_D} - {N_0})f,f) = ( Af,TAf) \geq c\|Af\|^2_{L^2(D;\,\mathbb{R}^2)} > 0 \quad \forall f\in L^2_0(\partial \Omega) \quad f\neq0.
 \end{equation}
